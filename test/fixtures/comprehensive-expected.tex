\documentclass{article}
\usepackage[utf8]{inputenc}
\usepackage[japanese]{babel}
\usepackage{amsmath}
\usepackage{array}

\title{包括的句読点テスト文書}
\author{テスト太郎}
\date{\today}

\begin{document}

\maketitle

\section{基本的な句読点テスト}

これは,基本的なテストです.また,これも,テストです.

\section{数字との組み合わせ}

2024年,日本では多くの研究が行われました.また,温度は25℃,湿度は60\%でした.
値段は1,000円,重量は2.5kgです.

\section{数式との組み合わせ}

数式 $x = y + z$ において,$x$ は結果を表します.
また,$E = mc^2$ という有名な式があります.

複雑な数式の場合:
\begin{equation}
\int_{-\infty}^{\infty} e^{-x^2} dx = \sqrt{\pi}
\end{equation}

この積分の結果は,$\sqrt{\pi}$ となります.

\section{英単語との組み合わせ}

プログラミング言語Python,JavaScript,TypeScriptなどがあります.
また,VS Code,GitHub,npmなどのツールも重要です.

データベースはMySQL,PostgreSQL,MongoDBが使われます.
APIはREST API,GraphQL APIなどがあります.

\section{表との組み合わせ}

以下の表は,プログラミング言語の特徴をまとめたものです.

\begin{table}[h]
\centering
\begin{tabular}{|l|l|l|}
\hline
言語 & 型システム & 用途 \\
\hline
Python & 動的型付け & データサイエンス,AI \\
TypeScript & 静的型付け & Webフロントエンド \\
Rust & 静的型付け & システムプログラミング \\
\hline
\end{tabular}
\caption{プログラミング言語の比較}
\end{table}

この表から,各言語の特徴が分かります.

\section{特殊文字との組み合わせ}

記号(),[],{}などと組み合わせた場合のテストです.
引用符"text",'text'との組み合わせもテストします.

URL https://example.com,メールアドレス test@example.com なども含みます.

\section{長い文章での連続テスト}

これは,非常に長い文章のテストです.句読点が多数含まれている場合,正しく変換されるかを確認します.たとえば,研究において,データの収集,分析,可視化,解釈という一連のプロセスがあります.また,機械学習では,前処理,特徴量エンジニアリング,モデル選択,評価という段階があります.さらに,深層学習では,ニューラルネットワークの設計,最適化,正則化,ハイパーパラメータチューニングが重要です.

\section{コードブロックとの組み合わせ}

以下は,Python のコード例です.

\begin{verbatim}
def process_data(data):
    # データ処理,変換,出力
    result = data.process()
    return result
\end{verbatim}

コード内の句読点は変換されません.しかし,説明文では,適切に変換されます.

\section{引用との組み合わせ}

「これは,引用文です.」という形式や,
『これも,引用文です.』という形式があります.

\section{箇条書きとの組み合わせ}

\begin{itemize}
\item 項目1:これは,最初の項目です.
\item 項目2:これは,2番目の項目です.
\item 項目3:これは,最後の項目です.
\end{itemize}

\begin{enumerate}
\item 手順1:まず,データを準備します.
\item 手順2:次に,分析を実行します.
\item 手順3:最後に,結果を解釈します.
\end{enumerate}

\section{結論}

このテストファイルにより,さまざまなパターンでの句読点変換が正しく動作することを確認できます.数字,数式,英単語,表,特殊文字など,実際のLaTeX文書で使われる要素との組み合わせをテストしました.

\end{document}